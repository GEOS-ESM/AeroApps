\section{File I/O}
%
To take care of the code I/O, the following modules are utilized.

\noindent
{\tt m\_stdio} defines standard I/O parameters (system dependent) in 
particular the I/O file unit numbers and the length (number of characters)
 of any file name.
%
\begin{verbatim}
EXAMPLE 1:
  Here are the actual parameter definitions contain in the modules.

     ! Standard i/o units.
     integer, parameter :: stdin  = 5
     integer, parameter :: stdout = 6

     #ifdef sysHP-UX        ! Special setting for HP-UX
        integer, parameter :: stderr = 7
     #else                  ! Generic setting for UNIX other than HP-UX
        integer, parameter :: stderr = 0
     #endif
     integer, parameter :: LEN_FILENAME = 128
\end{verbatim}
%
{\tt m\_ioutil} is a module containing several portable interfaces for
some highly system dependent, but frequently used I/O functions.
It performs operations such as opening and closing Fortran ``unformatted''
and ``formatted'' files regardless of the file options used.
It has the following subroutines.
%
\begin{itemize}
\item {\tt opniee:} Opens a file in `IEEE' format.
  	`IEEE' format is refered as a Fortran ``unformatted'' file with
  	``sequantial'' access and variable record lengths. Under common
  	Unix, it is only a file with records packed with a leading 4-byte
  	word and a trailing 4-byte word indicating the size of the record 
        in bytes. However, under UNICOS, it is also assumed to have 
        numerical data representations represented according to the IEEE 
        standard corresponding KIND conversions. Under a DEC machine, it 
        means that compilations of the source code should have the 
       ``-bigendian'' option specified.
\item {\tt clsieee:} Closes a logical unit opened by opnieee().
  	The reason for a paired clsieee() for opnieee() instead of a
  	simple close(), is for the portability reason. For example,
  	under UNICOS, special system calls may be need to set up the
  	unit right, and the status of the unit should be restored upon
  	close.
\item {\tt opntext:} Opens a text (ASCII) file. Under Fortran, it is 
      defined as ``formatted'' with ``sequential'' access.
\item {\tt clstext:} Closes a text file opend with an opntext() call.
\item {\tt luavail:} Looks for an available (not opened and not statically
      assigned to any I/O attributes to) logical unit.
\item {\tt luflush:} Calls available on many systems are often implementation
      dependent. This subroutine provides a uniform interface. It also 
      ignores invalid logical unit value. 
\end{itemize}
%
\begin{verbatim}
EXAMPLE 2:
  Code:
      use m_ioutil,only : luavail, opntext, clstext
      use m_ioutil,only : opnieee, clsieee
      integer :: unit1, unit2
      integer :: ier                        | to check possible error
      character*128 :: infile1='text_file.in'
      character*128 :: infile2='ieee_file.in'

      unit1 = luavail()
      call opntext(unit1,infile1,''unknown'',ier)
      ...
      unit2 = luavail()
      call opnieee(unit2,infile2,''unknown'',ier)
      ...
      call clstext(unit1,ier)
      call clsieee(unit2,ier)
\end{verbatim}
%
{\tt m\_FileResolv} provides routines for resolving GrADS-like file name 
templates.
The main subroutine is {\tt FileResolv:} that resolves file name templates, 
scp'ing files from remote and performing gunzip'ing as necessary
%
\begin{verbatim}
EXAMPLE 3:
  Code:
      use m_FileResolv,only : FileResolv
      character(len=*) :: expid          ! Experiment id
      integer          :: nymd           ! Year-month-day
      integer          :: nhms           ! Hour-min-sec
      character(len=*) :: templ          ! file name template
      character(len=*) :: fname          ! resolved file name (output)

      call FileResolv (expid, nymd, nhms, templ, fname)
            
\end{verbatim}
%
{\tt m\_mpout} is a multiple but mergable parallel output module that
performs operations such as opening/closing multiple files with the same 
name prefix, synchonizing and flushing the multiple output streams. 
The main operation are:
\begin{itemize}
\item {\tt mpout\_open}: Opens the multiple output streams.
\item {\tt mpout\_close}: Closes the multiple output streams.
\item {\tt mpout\_ison}: Verifies if "mpout" is properly defined by
      deciding if the current PE has a defined "mpout".
\end{itemize}
%
\begin{verbatim}
EXAMPLE 4:
   - mpout_ison() on no PE but PE 0, where mpout is opened to file
     "mpout.000":

       call mpout_open()

   - mpout_ison() on every 4 PE starting from 0, where mpout is opened
     to files named, "out.000", "out.004", "out.007", "out.00a", etc.

       call mpout_open(mask=3,pfix='out')

     Note that 3 = "11"b.  Therefore,

       mask="000011"b
       PE 0="000000"b  is "clean", mpout_ison()
          1="000001"b  is "dirty", .not. on
          2="000010"b  is "dirty", .not. on
          3="000011"b  is "dirty", .not. on
          4="000100"b  is "clean", mpout_ison()
          5="000101"b  is "dirty", .not. on
          6="000110"b  is "dirty", .not. on
          7="000111"b  is "dirty", .not. on
          8="001000"b  is "clean", mpout_ison()
          9="001001"b  is "dirty", .not. on
         10="001010"b  is "dirty", .not. on
         11="001011"b  is "dirty", .not. on
         12="001100"b  is "clean", mpout_ison()
         13="001101"b  is "dirty", .not. on
         14="001110"b  is "dirty", .not. on
         15="001111"b  is "dirty", .not. on
\end{verbatim}
%
